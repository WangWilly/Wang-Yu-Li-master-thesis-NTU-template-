% !TeX root = ./main.tex

% --------------------------------------------------
% 資訊設定(Information Configs)
% --------------------------------------------------

\ntusetup{
  university*   = {National Taiwan University},
  university    = {國立臺灣大學},
  college       = {電機資訊學院},
  college*      = {College of Electrical Engineering and Computer Science},
  institute     = {資訊工程學系},
  institute*    = {Department of Computer Science and Information Engineering},
  title         = {使用U-Net及其壓縮版本來進行歌聲分離},
  title*        = {Singing Voice Separation Using U-Net and Its Compressed Version},
  author        = {王俞禮},
  author*       = {Yu-Li Wang},
  ID            = {R08922181},
  advisor       = {張智星},
  advisor*      = {Jyh-Shing Roger Jang},
  date          = {2021-06-23},         % 若註解掉,則預設為當天
  oral-date     = {2020-06-23},         % 若註解掉,則預設為當天
  DOI           = {10.6342/NTU202102677},
  keywords      = {歌聲分離、U-Net、注意力模型、頻譜刪減、深度模型壓縮},
  keywords*     = {singing voice separation, U-Net, attention based model, spectrum subtraction, network compression},
}

% --------------------------------------------------
% 加載套件(Include Packages)
% --------------------------------------------------

\usepackage[sort&compress]{natbib}      % 參考文獻
\usepackage{amsmath, amsthm, amssymb}   % 數學環境
\usepackage{pgfplots}                   % 時間序列畫成趨勢圖
\usepackage{ulem, CJKulem}              % 下劃線、雙下劃線與波浪紋效果
\usepackage{booktabs}                   % 改善表格設置
\usepackage{multirow}                   % 合併儲存格
\usepackage{diagbox}                    % 插入表格反斜線
\usepackage{array}                      % 調整表格高度
\usepackage{longtable}                  % 支援跨頁長表格
\usepackage{paralist}                   % 列表環境



\usepackage{lipsum}                     % 英文亂字
\usepackage{zhlipsum}                   % 中文亂字

% --------------------------------------------------
% 套件設定(Packages Settings)
% --------------------------------------------------

\pgfplotsset{compat=1.17}               % 設定 pgfplots
\def\withcertification{1}