% !TeX root = ../main.tex

\begin{abstract}

歌聲分離領域英文稱 Singing voice separation,旨在將音樂中的「主唱音軌」與「伴奏音軌」分離出,可以在 time domain 或是 frequency domain 實現,後者是本研究的重點。深度學習已在現今聲音分離領域中是不可或缺的方法,本研究主要基於 Ronneberger 等人的 U-Net 架構,原用於分割生物醫學影像,並有很好的效果,本論文基於此架構改造,用於訓練頻譜圖的切割。模型的輸出為主唱(vocals stem)與伴奏(accompaniment stem)的頻譜,再基於 ratio mask filter 與 Wiener filter 理論,改善現有的 U-Net 模型;在注意力模型 attention gate 與 self-attention 的基礎上做延伸,設計一新的 U-Net 模型;基於先前頻譜刪減(spectral subtraction)的研究,探討其問題與設法提升其效果;使用模型剪枝(model pruning)與模型量化(model quantization)做模型壓縮。實驗使用到公開的資料集包含:MUSDB18、DSD100、MedleyDB、iKala,非公開的資料集包含:Ke(捷奏錄音室-柯老師)。本篇使用的是 Python 套件 Museval,由 SegSep 團隊維護而成,並在 2018 signal separation evaluation campaign 中所採取,其提供的 SDR 為實驗評估主要指標。(SDR 由 7.269 提高至 7.800)

\end{abstract}

\begin{abstract*}

Singing voice separation aims to separate the "vocal stem" and the "accompaniment stem" from the music. It can be implemented in time domain or frequency domain, the latter is the focus of this research. Deep learning has become a main method in the field of sound separation nowadays. The U-Net architecture that this research is mainly based on was proposed by Ronneberger et al. It was originally used to segment biomedical images. This paper is based on this architecture with improvement and is used to train the segmented spectrogram. The output of the model is the frequency spectrum of vocals stem and accompaniment stem.
Based on ratio mask filter and Wiener filter theory, improve the existing U-Net model. Design a new U-Net model based on attention gate and self-attention. Based on previous research on spectrogram subtraction, discuss its problems and try to improve its effect. Use model pruning and model quantization for model compression. The music data collection used includes public data collections, as well as data collections provided by the National Taiwan University MIRLAB. This study uses Museval to evaluate the gap between the "original audio track" and the "audio track predicted by the U-Net model". It is maintained by the SegSep team and adopted in the 2018 signal separation evaluation campaign. SDR is the main metric of experimental evaluation. (7.269 vs. 7.80 in SDR)

\end{abstract*}